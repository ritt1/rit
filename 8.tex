PROGRAM: A 

Instructions: How to Run?
•	Create a file name called track.php, copy the below code and paste it and save it.
•	Copy track.php file and open XAAMP directory if installed else install it click here
•	There you’ll find a folder named “htdocs”.
•	Inside the “htdocs” folder, paste track.php file.
•	After then open your XAAMP and start the Apache server.
•	Open your favorite browser; we recommend using Google Chrome or Mozilla Firefox.
•	Then, go to the URL “http://localhost/track.php“.

<?php
$counterFile = "counter.txt";
if (!file_exists($counterFile)) {
    file_put_contents($counterFile, "0");
}
$currentCount = file_get_contents($counterFile);
$newCount = $currentCount + 1;
file_put_contents($counterFile, $newCount);
?>
<!DOCTYPE html>
<html lang="en">
<head>
    <title>Visitor Counter | vtucode</title>
    <style>
        body {
            font-family: Arial, sans-serif;  
  text-align: center;
            margin: 0;
            padding: 0;
            display: flex;
            flex-direction: column;
            justify-content: center;
            height: 100vh;
            background-color: #f4f4f9;
            color: #333;
        }
        .container {
            background: #fff;
            padding: 20px;
            box-shadow: 0 2px 10px rgba(0, 0, 0, 0.1);
            border-radius: 8px;
            margin: 0 auto;
            width: 60%;
        }
        h1 {
            font-size: 2.5em;
            margin: 0;
        }
        p {
            font-size: 1.2em;
            color: #555;
        }
    </style>
</head>

<body>
    <div class="container">
        <h1>Welcome to Our Website!</h1>
        <p>You are visitor number: <strong><?php echo $newCount; ?></strong></p>
    </div>
</body>
</html>


PROGRAM: B

<?php
$servername = "localhost";
$username = "root";
$password = "";
$dbname = "students";
$conn = new mysqli($servername, $username, $password, $dbname);
if ($conn->connect_error) {
    die("Connection failed: " . $conn->connect_error);
}
$sql = "SELECT * FROM students";
$result = $conn->query($sql);

$students = [];
if ($result->num_rows > 0) {
    while ($row = $result->fetch_assoc()) {
        $students[] = $row;
    }
}
function selectionSort(&$arr, $key)
{
    $n = count($arr);
    for ($i = 0; $i < $n - 1; $i++) {
        $minIndex = $i;
        for ($j = $i + 1; $j < $n; $j++) {
            if ($arr[$j][$key] < $arr[$minIndex][$key]) {
                $minIndex = $j;
            }
        }
        $temp = $arr[$i];
        $arr[$i] = $arr[$minIndex];
        $arr[$minIndex] = $temp;
    }
}
selectionSort($students, 'name');
?>
<!DOCTYPE html>
<head>
    <title>Sorted Student Records | vtucode</title>
    <style>
        body {
            font-family: 'Segoe UI', Tahoma, Geneva, Verdana, sans-serif;
            background-color: #f0f2f5;
            color: #333;
            margin: 0;
            padding: 20px;
        }
        h2 {
            text-align: center;
            color: #4A90E2;
            margin-bottom: 20px;
        }
        table {
            width: 100%;
            border-collapse: collapse;
            background-color: #fff;
            border-radius: 10px;
            overflow: hidden;
            box-shadow: 0 2px 8px rgba(0, 0, 0, 0.1);
            margin: 0 auto;
        }
        th,
        td {
            padding: 12px 15px;
            text-align: left;
            border-bottom: 1px solid #ddd;
        }
        th {
            background-color: #4A90E2;
            color: white;
            text-transform: uppercase;
            letter-spacing: 0.03em;
        }
        tr {
            transition: background-color 0.3s ease;
        }
        tr:hover {
            background-color: #f1f1f1;
        }
        td {
            font-size: 0.9em;
            color: #555;
        }
        @media (max-width: 768px) {
            table,
            th,
            td {
                display: block;
                width: 100%;
            }
            th,
            td {
                box-sizing: border-box;
            }
            tr {
                margin-bottom: 15px;
                display: block;
                box-shadow: 0 2px 5px rgba(0, 0, 0, 0.1);
            }
            th {
                position: absolute;
                top: -9999px;
                left: -9999px;
            }
            td {
                border: none;
                position: relative;
                padding-left: 50%;
                text-align: right;
            }
            td:before {
                content: attr(data-label);
                position: absolute;
                left: 0;
                width: 50%;
                padding-left: 15px;
                font-weight: bold;
                text-align: left;
                text-transform: uppercase;
                color: #4A90E2;
            }
        }
    </style>
</head>
<body>
    <h2>Sorted Student Records by Name</h2>
    <table>
        <thead>
            <tr>
                <th>ID</th>
                <th>Name</th>
                <th>USN</th>
                <th>Branch</th>
                <th>Email</th>
                <th>Address</th>
            </tr>
        </thead>
        <tbody>
            <?php foreach ($students as $student): ?>
                <tr>
                    <td data-label="ID"><?php echo htmlspecialchars($student['id']); ?></td>
                    <td data-label="Name"><?php echo htmlspecialchars($student['name']); ?></td>
                    <td data-label="USN"><?php echo htmlspecialchars($student['usn']); ?></td>
                    <td data-label="Branch"><?php echo htmlspecialchars($student['branch']); ?></td>
                    <td data-label="Email"><?php echo htmlspecialchars($student['email']); ?></td>
                    <td data-label="Address"><?php echo htmlspecialchars($student['address']); ?></td>
                </tr>
            <?php endforeach; ?>
        </tbody>
    </table>
</body>
</html>
