#define INF 999 
#define MAX 100 
int p[MAX], c[MAX][MAX], t[MAX][2]; 
int find(int v) 
{ 
while (p[v]) 
v = p[v]; 
return v; 
} 
void union1(int i, int j) 
{ 
p[j] = i; 
} 
void kruskal(int n) 
{ 
int i, j, k, u, v, min, res1, res2, sum = 0; 
for (k = 1; k < n; k++) 
{ 
min = INF; 
for (i = 1; i < n - 1; i++) 
{ 
for (j = 1; j <= n; j++) 
{ 
if (i == j) continue; 
if (c[i][j] < min) 
{ 
u = find(i); 
v = find(j); 
if (u != v) 
{ 
res1 = i; 
res2 = j; 
min = c[i][j]; 
} 
} 
} 
} 
union1(res1, find(res2)); 
t[k][1] = res1; 
t[k][2] = res2; 
sum = sum + min; 
} 
printf("\nCost of spanning tree is=%d", sum); 
printf("\nEdgesof spanning tree are:\n"); 
for (i = 1; i < n; i++) 
printf("%d -> %d\n", t[i][1], t[i][2]); 
} 
int main() 
{ 
int i, j, n; 
printf("\nEnter the n value:"); 
scanf("%d", & n); 
for (i = 1; i <= n; i++) 
p[i] = 0; 
printf("\nEnter the graph data:\n"); 
for (i = 1; i <= n; i++) 
for (j = 1; j <= n; j++) 
scanf("%d", & c[i][j]); 
kruskal(n); 
return 0; 
} 
