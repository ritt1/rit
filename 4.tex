#include<stdio.h> 
#define INF 999  
void dijkstra(int c[10][10],int n,int s,int d[10]) 
{ 
    int v[10],min,u,i,j; 
    for(i=1; i<=n; i++) 
    { 
        d[i]=c[s][i]; 
        v[i]=0; 
    } 
    v[s]=1; 
    for(i=1; i<=n; i++) 
    { 
        min=INF; 
        for(j=1; j<=n; j++) 
            if(v[j]==0 && d[j]<min) 
            { 
                min=d[j]; 
                u=j; 
            } 
        v[u]=1; 
        for(j=1; j<=n; j++) 
            if(v[j]==0 && (d[u]+c[u][j])<d[j]) 
                d[j]=d[u]+c[u][j]; 
    } 
} 
int main() 
{ 
    int c[10][10],d[10],i,j,s,sum,n; 
    printf("\nEnter n value:"); 
    scanf("%d",&n); 
    printf("\nEnter the graph data:\n"); 
    for(i=1; i<=n; i++) 
        for(j=1; j<=n; j++) 
            scanf("%d",&c[i][j]); 
    printf("\nEnter the souce node:"); 
    scanf("%d",&s); 
    dijkstra(c,n,s,d); 
    for(i=1; i<=n; i++) 
        printf("\nShortest distance from %d to %d is %d",s,i,d[i]); 
    return 0; 
} 
